\problemname{Hopper}

A hopper is a virtual creature that visits Java programs and explores their arrays. Scientists
observed a hopper and came to the following conclusions:
\begin{itemize}
\item a hopper only visits arrays with integer entries, 
\item a hopper always explores a sequence of array elements using the
  following rules:
  \begin{itemize}
  \item a hopper cannot jump too far, that is, the next element is
    always at most $D$ indices away (how far a hopper can jump depends
    on the length of its legs),
  \item a hopper doesn’t like big changes in values---the next element
    differs from the current element by at most $M$, more precisely
    the absolute value of the difference is at most $M$ (how big a
    change in values a hopper can handle depends on the length of its
    arms), and
  \item a hopper never visits the same element twice.
  \end{itemize}
\item a hopper will explore the array with the longest exploration
  sequence.
\end{itemize}
The scientists now need to prepare arrays for further study of hoppers and they need your
help. They want a program that given an array and values $D$ and $M$ computes the length
of the longest exploration sequence in the array.

\section*{Input}

The first line contains three numbers $n$, $D$,
$M$, where $n$ is the length of the array (as described above, $D$ is
the maximum length of a jump the hopper can make, and $M$ is the
maximum difference in values a hopper can handle). The next line
contains $n$ integers---the entries of the array. We have
$1 \leq D \leq 7$, $1 \leq M \leq 10\,000$, $1 \leq n \leq 10\,000$
and the integers in the array are between $-1\,000\,000$ and
$1\,000\,000$.

\section*{Output}

The output contains one line---the length of the
longest exploration sequence. The hopper can start and end at any
location and the length is computed as the number of visited
locations.
